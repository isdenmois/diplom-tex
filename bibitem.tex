\begin{thebibliography}{30}
\bibitem{cordova:android_sdk} Download Android Studio and SDK Tools. [Электронный ресурс], Android Developers, URL: https://developer.android.com/intl/ru/sdk/index.html, [Дата обращения: 6 июня 2015].

\bibitem{cordova:java_sdk} Java SE - Downloads. [Электронный ресурс], Oracle Technology Network, URL: http://www.oracle.com/technetwork/java/javase/downloads/index.html, [Дата обращения: 6 июня 2015].

\bibitem{android:publish} Android Platform Guide. [Электронный ресурс],  Официальный сайт Apache Cordova. URL: http://cordova.apache.org/docs/en/5.0.0/guide\_platforms\_android\_index.md.html, [Дата обращения: 6 июня 2015].

\bibitem{mobile_count} Список стран по числу используемых мобильных телефонов. [Электронный ресурс], Электронная энциклопедия Wikipedia, URL: https://ru.wikipedia.org/wiki/Список\_стран\_по\_числу\_используемых\_моби- льных\_телефонов, [Дата обращения: 13 мая 2015].

\bibitem{cordova} Apache Cordova --- фреймворк для создания мобильных приложений [Электронный ресурс], Официальный сайт Apache Cordova, URL: http://cordova.apache.org, [Дата обращения: 13 мая 2015].

\bibitem{backbone} Backbone.js по-русски . [Электронный ресурс], Официальный сайт Backbone.js, URL: http://backbonejs.ru, [Дата обращения: 19 июня 2015].

% \bibitem{QA2} Cooper R., Ruger S. A Simple Question Answering System. New York, 2007.
% \bibitem{QA} Gaizauskas R., Humphreys K. A Combined IR/NLP Approach to Question Answering Against Large Text Collection. Sheffield, 2010.
% \bibitem{16} Gruber T. R. A translation approach to portable ontologies. USA, 1993.
% \bibitem{QAS} Guda V., Sanamrudi S. K. Approaches for question answering. International Journal of Engineering Science and technology, 2011-3
% \bibitem{18} Авдошин С.М., Шатилов М.П. Информационные технологии онтологического инжиниринга. M, 2008
\bibitem{web1} Акимов С.В. Технологии Internet. // СПб, 2005
% \bibitem{17} Гладун А.Я., Рогушина Ю.В. Онтологии в корпоративных системах. Оренбург, 2006
% \bibitem{frame} Горбунов-Посадов М. М. Расширяемые программы. М, 1999
\bibitem{VBS} Грошев А.С. программирование на VBS. // М, 2007
\bibitem{perl} Маслов В.В. Введение в Perl. // Оренбург, 2000
% \bibitem{19} Овдей О.М., Проскудина Г.Ю. Обзор инструментов инженерии онтологий. М, 2007
\bibitem{php} Томпсон Л., Веллинг Л. Разработка Web-приложений на PHP и MySQL. // Спб, 2008
% \bibitem{web2} Храмцов П.Б., Брик С.А., Русак А.М., Сурин А.И. Основы web-технологий. М, 2008
% \bibitem{QA1} Hirschman L., Gaizauskas R. Natural language question answering: view from here. USA, 2001
% \bibitem{DB} DBpedia Mappings Wiki URL: http://mappings.dbpedia.org/index.php (дата обращения: 11.05.2015)
% \bibitem{start} START: Natural Language Question Answering System  URL: http://start.csail.mit.edu/index.php (дата обращения: 13.05.2015)
% \bibitem{python} Разработка сайтов и приложений на Python URL: http://secl.com.ua/python.html (дата обращения: 13.05.2015)
% \bibitem{ask} Семантическая поисковая система AskNet URL: http://asknet.ru (дата обращения: 12.05.2015)
% \bibitem{ephyra} Open Source Question Answering Frameworks URL: http://www.ephyra.info/ (дата обращения: 13.05.2015)
% \bibitem{py_fram} WebFrameworks URL: https://wiki.python.org/moin/WebFrameworks (дата обращения: 13.05.2015)
% \bibitem{watson} What is Watson? 
% 
% URL: http://www.ibm.com/smarterplanet/us/en/ibmwatson/what-is-watson.html (дата обращения: 13.05.2015)
% \bibitem{dbpedia} DBpedia  URL: http://dbpedia.org/About (дата обращения: 11.05.2015)
% \bibitem{WA} WolframAlpha. Computation Knowledge Engine URL: http://www.wolframalpha.com/ (дата обращения: 13.05.2015)
% 
% URL: http://www.w3.org/blog/SW/2008/01/15/sparql\_is\_a\_recommendation/ (дата обращения: 12.05.2015)
% \bibitem{SP1} SPARQL Query Language for RDF URL: http://www.w3.org/TR/rdf-sparql-query/ (дата обращения: 13.05.2015)
% 
\end{thebibliography}
% \clearpage
% \bibliographystyle{unsrt}
% \bibliography{bibs}{}