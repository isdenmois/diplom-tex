\frontmatter{Введение}
В связи с бурным развитием информационных технологий и непрерывным увеличением объемов информации, доступной в глобальной сети Интернет, всё большую актуальность приобретают вопросы эффективного построения пользовательских интерфейсов. Развиваются не только компьютеры, но и сети. Еще несколько десятков лет Интернет представлял собой небольшую частную сеть, но теперь это миллиарды устройств и система, занимающая все большую часть в современной жизни.

Веб-технологии изменили представление о работе с информацией. Оказалось, что традиционные параметры развития вычислительной техники --- такие как производительность, емкость запоминающих устройств --- не учитывали основного узкого места --- интерфейса для взаимодействия с человеком.

Веб-приложения представляют собой особый тип программ, построенных по архитектуре <<клиент-сервер>>. Основные вычисления происходят на сервере, а клиент отвечает за взаимодействие с пользователем. Однако, технологии развиваются и на смену статичным страницам пришли интерактивные приложения с использованием JavaScript и jQuery. В настоящее время все популярнее становятся фреймворки, реализующие паттерн MVC. С их помощью логика клиентского приложения отделяется от представления, происходит стандартизация кодирования и увеличивает эффективность разработки.

Одним из основных инструментов для создания современных веб-приложений является проект Angular.JS, разработанный и сопровождаемый компанией Google. Этот фреймворк имеет достаточно низкий порог вхождений, использует декларативный подход, отделяет логику от представления и позволяет создавать одностраничные приложения, в которых представления меняются без перезагрузки страницы. При этом взаимодействие с сервером происходит в асинхронном режиме.

Также нельзя обойти стороной мобильные устройства. По состоянию на начало 2015 года количество активных мобильных устройств уже давно превысило количество людей на планете\cite{mobile_count}. В связи с этим повышается необходимость разработки мобильного кросс-платформенного приложения. Самым известным и используемым инструментом является Apache Cordova\cite{cordova}. Данный фреймворк позволяет создавать универсальные мобильные приложения, работающие на различных мобильных платформах, с использованием стандартных веб-технологий.

Целью данной работы является реализация кроссплатформенного приложения программной системы вероятностного моделирования ad-hoc сетей в рамках веб-технологий. Для достижения данной цели были поставлены следующие задачи:
\begin{enumerate}
 \item Изучить проблематику и примеры веб-инструментов в реализованных веб-приложениях.
 \item Произвести обзор веб-технологий, позволяющих создавать эффективные и удобные веб-приложения с использованием шаблонов проектирования.
 \item Реализовать свое веб-приложение, взаимодействующее с веб-сервером по протоколу SOAP и позволяющее производить вероятностное моделирование ad-hoc сетей.
 \item Реализовать приложение с использованием веб-технологий, функционирующие на широком спектре мобильных платформ.
\end{enumerate}

\clearpage