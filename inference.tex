\frontmatter{Заключение}

В рамках данной выпускной квалификационной работы бакалавра были рассмотрены основные способы создания эффективных и удобных веб-приложений с использованием шаблонов проектирования. Также были изучены преимущества и недостатки различных инструментов как с теоритической так и с практической стороны.

Пользуясь полученными знаниями, было разработано кроссплатформенное приложение для системы вероятностного моделирования ad-hoc сетей с использованием Backbone.js. Затем оно было адаптировано для использования с AngularJS. Далее, это приложение было развернуто для запуска на широком спектре мобильных устройств с использованием Apache Cordova. Для обмена данных с сервисом был использован протокол SOAP.

Также в рамках данной работы было продемонстрировано использование утилит для автоматизации сборки и разворачивания приложения. Был подробно описан процесс сборки, подписи и демонстрация работы на платформе Android.

Актуальность данной работы заключается в том, что постепенно влияние сети Интернет возрастает, многие приложения получают веб-реализацию и необходимо выдерживать качество кодирования. Также количество платформ с каждым годом увеличивается и, следовательно, возрастает потребность в кроссплатформенных средствах разработки, позволяющих одному коду работать как в различных операционных системах, так и на различных устройствах с различными размерами экрана. Приложение, реализованное в данной работе отражает все преимущества кроссплатформенной и веб-разработки с использованием шаблонов проектирования.