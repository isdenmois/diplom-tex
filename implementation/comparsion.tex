\section{Выбор инструментов}
На этапе проектирования приложения было рассмотрено несколько инструментов для создания одностраничных приложений с использованием шаблона проектирования MVC. Во внимание были приняты AngularJS, Backbone.js, Ember.js, Meteor и CanJS. Эти фреймворки были рассмотрены с теоретической стороны, были выявлены их достоинства и недостатки.

Meteor предлагает интересную концепцию -- общий код клиента и сервера, но он пока еще только в процессе разработки и пока рано делать какие-либо выводы. Самым интересными интересными и подающими надежды оказались AngularJS и Ember.js. Концепты Ember.js сложны в освоении и понимании. И, хотя официальная документация описывает все аспекты данного фреймворка, является актуальной и точной, но ей не хватает концепции. Для начинающего разработчика одностраничных приложений достаточно тяжело увидеть общую картину и понять архитектуру при использовании этого инструмента.

При начальном рассмотрении задачи было решено использовать для реализации фреймворк Backbone.js. В процессе выполнения задачи были выявлены достоинства и недостатки этого инструмента с практической стороны. После чего приложение было реализовано с использованием AngularJS. На практике были подтверждены удобства двунаправленного связывания, вложенных представлений и представления связываний. Затем, на основе последнего приложения были созданы отдельные шаблоны представлений для мобильных устройств. Далее, с использованием Apache Cordova было собрано мобильное приложение, которое запускается нативно на широком спектре устройств.