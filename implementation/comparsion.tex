\section{Практическое сравнение AngularJS и Backbone.js}

Таким образом на практике были рассмотрены преимущества и недостатки двух инструментов. Действительно, оказалось, что Backbone.js дает более быстрый старт, но с увеличением размера приложения код становится все запутаннее и необходимо иметь немалый опыт в проектировании приложений прежде чем использовать этот фреймворк. Также для разделения представлений приходится либо использовать существующий либо писать свой шаблонизатор. Но такой подход дает и свои преимущества: разработчик сам решает как структурировать свой код, какими воспользоваться компонентами.

Использование AngularJS дает больше преимуществ:
\begin{enumerate}
 \item Использование HTML-директив. Декларативный стиль описания шаблонов улучшает читаемость кода, упрощает разработку и разделяет логику от представления.
 \item Двухстороннее связывание. Позволяет автоматически обновлять представление при изменении модели.
 \item Вложенные представления. Позволяют вынести шаблоны в отдельные файлы
 \item Удобный механизм маршрутизации. Позволяет задавать как контроллеры так и текущее представление.
 \item Меньший размер библиотеки (39 Кб против 45 Кб).
 \item Размер сообщества на порядок выше\cite{angular}.
\end{enumerate}

С другой стороны, Backbone.js работает быстрее, имеет более долгую историю развития, сформированные концепции и более прост в освоении.