На этапе проектирования приложения была проведена декомпозиция на три логические части:

\begin{enumerate}
 \item Построение нового задания для создания исследования.
 \item Отображение списка поставленных исследований.
 \item Вывод подробной информации о проведенном исследовании.
\end{enumerate}

Для каждой части приложения были сверстаны свои шаблоны, описаны модели и закодированы контроллеры.

На этапе построения нового задания клиентское приложение запрашивает у веб-сервиса список доступных алгоритмов для постановки в очередь заданий. Также для каждого алгоритма получается список необходимых параметров. После чего из шаблона формируется форма со списком доступных алгоритмов. При смене алгоритма меняется и набор параметров. После заполнения полей формы, происходит валидация выбранных значений и отправляется запрос на другой веб-сервис. Этот веб-сервис проводит повторную валидацию полученных данных, связывается с менеджером базы данных и перенаправляет запрос в очередь менеджера. Далее сервис возвращает статус, отображающий успех постановки в очередь и идентификатор запроса. Если ответ приходит с ошибкой, то эта ошибка отображается. При успешном статусе, происходит смена представления: отображается список ранее поставленных заданий для проведения исследований.

В другом представлении происходит отображение списка ранее поставленных задач для исследований. Для каждого задания указывается статус готовности. Вся необходимая информация получается с третьего веб-сервиса, который запрашивает у менеджера базы данных список текущих заданий в очереди, список проведенных исследовании и проводимое в данный момент исследование. На основании этих данных по заданному шаблону представления формируется список и выводится пользователю.

Те исследования, что уже были проведены, доступны для подробного изучения. У различных исследований результаты зависят от поставленных алгоритмов и могут различаться. Приложение посылает идентификатор проведенного исследования на веб-сервис и ожидает результат вместе с его описанием. Результат может даже иметь разный вид. Это могут быть массивы точек, по которым необходимо построить график, или же некоторые статистические характеристики, которые ожидал увидеть пользователь. После обработки описания формата происходит заполнение определенного шаблона представления данными результата и выводится пользователю.