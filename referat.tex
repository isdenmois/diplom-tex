\documentclass[utf8,usehyperref,14pt,final]{G7-32}
\usepackage{graphicx}
\usepackage{textcase}
\graphicspath{{images/}}
\usepackage[labelsep=endash]{caption}
\captionsetup[table]{singlelinecheck=false} %, margin=5mm}

% \usepackage{setspace}
% \onehalfspacing
% \usepackage[onehalfspacing]{setspace}
% \usepackage[T2A]{fontenc} % кодировка шрифта
% \usepackage[russian,english]{babel}
\usepackage{cyrtimes}
%\usepackage{float,amsmath}
%\usepackage{listings}
%Отступы у страниц
\usepackage{indentfirst}
\renewcommand{\rmdefault}{ftm} % Times New Roman
\setlength{\parindent}{10mm}
\sloppy %переносить длинные слова
%\usepackage[colorlinks=false]{hyperref}
\renewcommand{\thesubsection}{\t\arabic{chapter}}
%\newcommand{\Section}[1][2]{
%	\section*{#1}\addcontentsline{toc}{section}{#2}
%	\centering{#2}
%}
\newcommand{\Section}[1]{
%	\vspace{4mm}
	#1\addcontentsline{toc}{section}{#1}
%	\vspace{4mm}
	
}
\newcommand{\sEction}[1]{
	\vspace{4mm}
	\indent#1
	\addcontentsline{toc}{section}{{~~}#1}
	\vspace{4mm}
}
\newcommand{\SEction}[1]{
	\vspace{4mm}
	\indent#1
	\addcontentsline{toc}{section}{{~~~~}#1}
	\vspace{4mm}
}

\newcommand{\Chapter}[1]{
\centerline{\MakeTextUppercase{#1}}\addcontentsline{toc}{section}{#1}
\vspace{8mm}
}

\newcommand{\ChapteR}[1]{
	\line{#1}\addcontentsline{toc}{section}{#1}
	\vspace{8mm}
}

\newcommand{\Intro}[1]{
\centerline{#1}\addcontentsline{toc}{section}{#1}{\vspace{8mm}#1}
\vspace{8mm}
}
%\newcommand\contentsname{\centerline{СОДЕРЖАНИЕ}{~~~}}
%\renewcommand*{\thesection}{\arabic {section}}
%\include{listings.inc}
\begin{document}
\frontmatter{РЕФЕРАТ}

Работа посвящена разработке и реализации веб-клиента как части системы исследования моделей ad hoc сетей. 
Целью данной работы является реализация кроссплатформенного приложения программной системы вероятностного моделирования ad-hoc сетей в рамках веб-технологий. Для достижения данной цели были поставлены следующие задачи:

\begin{enumerate}

\item Изучить проблематику и примеры веб-инструментов в реализованных
 веб-приложениях.

\item Произвести обзор веб-технологий, позволяющих создавать эффективные и удобные веб-приложения с использованием шаблонов проектирования.

\item Реализовать свое веб-приложение, взаимодействующее с веб-сервером
 по протоколу SOAP и позволяющее производить вероятностное моделирование ad-hoc сетей.

\item Реализовать приложение с использованием веб-технологий, функционирующие на широком спектре мобильных платформ.

\end{enumerate}

Работа состоит из 65 страниц текста, разделенного на введение, четыре главы, список из 16 использованных источников, заключение и три приложения, содержащие программный код. В работе представленно 20 рисунков и приведены примеры программного кода. Первая половина текста содержит обзор используемых автором технологий и программных средств, обоснование их выбора, а также вопросы адаптации этих средств для решения поставленной задачи. В третьей главе описано созданное одностраничное приложение с использованием Backbone.js и AngularJS. Четвертая глава посвящена описанию реализации мобильного приложения.

\end{document}
