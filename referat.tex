\documentclass[utf8,usehyperref,14pt,final]{G7-32}
\usepackage{graphicx}
\usepackage{textcase}
\graphicspath{{pics/}}
\usepackage[labelsep=endash]{caption}
\captionsetup[table]{singlelinecheck=false} %, margin=5mm}

\usepackage{setspace}
\onehalfspacing
\usepackage[T2A]{fontenc} % кодировка шрифта
\usepackage[russian,english]{babel}
\usepackage{cyrtimes}
%Отступы у страниц
\usepackage{indentfirst}
\renewcommand{\rmdefault}{ftm} % Times New Roman
\setlength{\parindent}{10mm}
\sloppy %переносить длинные слова

\renewcommand{\frontmatter}[1]{
    \clearpage
    \centerline{\MakeTextUppercase{#1}}
    \addcontentsline{toc}{chapter}{#1}
    \vspace{8mm}
}
\renewcommand{\backmatter}[2]{
    \clearpage
    \centerline{\MakeTextUppercase{#1}}
    \centerline{#2}
    \addcontentsline{toc}{chapter}{#1 #2}
    \vspace{8mm}
}
\newcommand\contentsname{\centerline{СОДЕРЖАНИЕ}{~~~}}
\usepackage{listings}

\lstdefinelanguage{JavaScript}{
  keywords={break, case, catch, continue, debugger, default, delete, do, else, finally, for, function, if, in, instanceof, new, return, switch, this, throw, try, typeof, var, void, while, with},
  morecomment=[l]{//},
  morecomment=[s]{/*}{*/},
  morestring=[b]',
  morestring=[b]",
  sensitive=true
}

\lstset{ %
  language=HTML,                 % выбор языка для подсветки (здесь это С)
  basicstyle=\small, % размер и начертание шрифта для подсветки кода
  extendedchars=\true,
  showspaces=false,            % показывать или нет пробелы специальными отступами
  showstringspaces=false,      % показывать или нет пробелы в строках
  showtabs=false,             % показывать или нет табуляцию в строках
  tabsize=2,                 % размер табуляции по умолчанию равен 2 пробелам
  breaklines=true,           % автоматически переносить строки (да\нет)
  breakatwhitespace=false, % переносить строки только если есть пробел
%   frame=single
}

\begin{document}
\frontmatter{РЕФЕРАТ}

Работа посвящена разработке и реализации веб-клиента как части системы исследования моделей ad hoc сетей. 
Целью данной работы является реализация кроссплатформенного приложения программной системы вероятностного моделирования ad-hoc сетей в рамках веб-технологий. Для достижения данной цели были поставлены следующие задачи:

\begin{enumerate}

\item Изучить проблематику и примеры веб-инструментов в реализованных
 веб-приложениях.

\item Произвести обзор веб-технологий, позволяющих создавать эффективные и удобные веб-приложения с использованием шаблонов проектирования.

\item Реализовать свое веб-приложение, взаимодействующее с веб-сервером
 по протоколу SOAP и позволяющее производить вероятностное моделирование ad-hoc сетей.

\item Реализовать приложение с использованием веб-технологий, функционирующие на широком спектре мобильных платформ.

\end{enumerate}

Работа состоит из 65 страниц текста, разделенного на введение, четыре главы, список из 16 использованных источников, заключение и три приложения, содержащие программный код. В работе представленно 20 рисунков и приведены примеры программного кода. Первая половина текста содержит обзор используемых автором технологий и программных средств, обоснование их выбора, а также вопросы адаптации этих средств для решения поставленной задачи. В третьей главе описано созданное одностраничное приложение с использованием Backbone.js и AngularJS. Четвертая глава посвящена описанию реализации мобильного приложения.

\end{document}
